\section*{Problem Statement}

Despite increasing awareness of the global mental health crisis, existing therapeutic models remain limited in their ability to respond to the scale, urgency, and complexity of need. 

Traditional talk therapy—while deeply effective in the context of a strong therapeutic alliance—often unfolds in weekly or biweekly sessions, creating large gaps between touchpoints. During those gaps, clients may experience emotional dysregulation, relapse triggers, or crisis moments without structured support or reflection tools. For therapists, a lack of visibility into a client’s lived experience between sessions can delay insight, disrupt continuity, and reduce the effectiveness of care.

Meanwhile, clinical systems face systemic challenges:

\begin{itemize}
  \item Overextended providers with high caseloads and burnout
  \item Fragmented data from EHRs, journaling apps, and self-report tools
  \item Ethical and privacy concerns around digital tracking of mental health data
  \item A lack of interoperable, open-source tools designed with therapeutic workflows in mind
\end{itemize}

These limitations are especially acute in communities disproportionately impacted by trauma, substance use, economic hardship, or limited access to care. Technology can help—but only if it's designed in close alignment with the values of privacy, dignity, and human-centered practice.

There is a critical need for a system that:
\begin{itemize}
  \item Supports continuity between therapy sessions
  \item Respects the client’s autonomy, privacy, and lived context
  \item Assists clinicians—not replaces them—with meaningful insight into emotional trends and reflection patterns
\end{itemize}

Without such a system, therapy risks remaining siloed, episodic, and outpaced by the real-time emotional demands placed on clients and providers alike.
