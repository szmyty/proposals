\section{Proposed Solution}

\textbf{THERAPY} is a modular, AI-augmented ecosystem designed to strengthen the therapeutic relationship between clients and clinicians through seamless, secure, and meaningful data exchange between sessions.

At its core, the system provides:

\begin{itemize}
  \item \textbf{A client-facing mobile application}  
    for guided journaling, mood tracking, and personalized emotional insight — with prompts designed to support self-awareness and emotional literacy.

  \item \textbf{An AI-assisted backend}  
    that summarizes emotional patterns, detects recurring cognitive distortions, and highlights themes — all ethically scoped, locally contextualized, and privacy-aware.

  \item \textbf{A therapist-facing dashboard}  
    to surface high-level trends, potential red flags, and narrative context — helping clinicians enter each session with deeper understanding, without extra burden.
\end{itemize}

\bigskip

\noindent The system operates as a closed feedback loop:

\begin{enumerate}
  \item Clients reflect and document their experiences throughout the week
  \item The AI backend processes entries and mood logs into structured, summarized insights
  \item Therapists review these summaries prior to sessions or at check-in intervals
  \item Insights inform the therapeutic process and adapt future prompts accordingly
\end{enumerate}

\bigskip

\noindent This loop is designed to preserve the sanctity of the therapist-client relationship, while amplifying its impact through context, memory support, and deeper emotional pattern recognition.

\bigskip

Unlike mental health chatbots or consumer wellness trackers, THERAPY is built not to replace human care — but to enhance it. It assumes that people already know something is wrong; the problem is often knowing how to understand it, track it, and bring it into a healing space.

\bigskip

\textbf{THERAPY is that space — extended, reinforced, and always listening.}
