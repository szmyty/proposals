\section*{Proposed Solution}

To address the disconnect between therapeutic intention and lived emotional experience, we propose the development of \textbf{THERAPY} — a modular, open-source digital ecosystem that augments the therapeutic process with continuous, ethically guided support.

THERAPY is not a replacement for human care — it is a companion to it. The system is designed to maintain the centrality of the therapist-client relationship while introducing technology to enhance reflection, insight, and continuity.

At its core, the platform integrates three primary components:

\begin{enumerate}
  \item \textbf{Client Tools} — Daily journaling prompts, mood tracking, and guided reflection tools powered by AI summarization. These features help clients externalize and organize their thoughts between sessions without needing to over-share raw data.
  
  \item \textbf{Therapist Dashboard} — A secure, low-burden interface that surfaces high-level emotional trend summaries, journaling themes, and mood shifts over time. It enables therapists to gain insight without increasing administrative load or violating trust.
  
  \item \textbf{Ethical Infrastructure} — Built-in privacy controls, transparent AI behavior, and customizable data policies that keep the client in control. All features are designed in alignment with trauma-informed care and ethical AI principles.
\end{enumerate}

This approach offers a lightweight, extensible system for maintaining connection, enhancing awareness, and enabling earlier intervention. It is especially well-suited for high-stress or resource-limited environments where traditional care models are strained.

THERAPY can be deployed in both clinical and community contexts, with the flexibility to serve as:

\begin{itemize}
  \item A companion to ongoing therapy
  \item A structured journaling and reflection tool for individuals on waiting lists or in between providers
  \item A data-cooperative framework for researchers studying mental health patterns with client-informed consent
\end{itemize}

By combining human empathy with ethical automation, THERAPY represents a new layer of continuity in mental health care — one that is scalable, compassionate, and clinically informed.
