
\section{Research Context and Related Findings}

Recent literature supports the connection between under-resourced mental health systems, rising suicide rates, and the opioid epidemic — particularly with fentanyl.

\subsection*{Key Study Highlights}

\begin{itemize}
\item \textbf{Can buprenorphine reduce suicide mortality? (L. Sher et al., 2022)}\\
    Buprenorphine treatment for opioid use disorder (OUD) in Veterans Affairs patients showed a lower suicide mortality rate compared to methadone treatment.\\
    \href{https://doi.org/10.1111/acps.13510}{DOI:10.1111/acps.13510}
    
\item \textbf{Mental health impact of fentanyl abuse, a case report (G. L. Chapatte et al., 2024)}\\
    Fentanyl abuse has become a serious public health issue mainly affecting young people, with high levels of impulsivity and lack of frustration tolerance predisposing to illicit substance use.\\
    \href{https://doi.org/10.1192/j.eurpsy.2024.843}{DOI:10.1192/j.eurpsy.2024.843}
    
\item \textbf{Exploring the Psychological Side of Fentanyl: A Scoping Review to Disclose the Psychosocial Dimensions of Illicitly Manufactured Fentanyl Users. (Pasquale Caponnetto et al., 2024)}\\
    Illicitly Manufactured Fentanyl users are younger, have low risk perception, and are significantly associated with mental disorders like suicidal thoughts, anxiety disorders, and depression.\\
    \href{https://doi.org/10.52965/001c.120958}{DOI:10.52965/001c.120958}
    
\end{itemize}

\noindent These studies reinforce the need for systems like \textbf{THERAPY} that bridge the gap between daily emotional experiences and structured clinical care.
